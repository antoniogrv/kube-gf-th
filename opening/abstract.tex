% 
%            ,,                                        
%          `7MM            _.o9                                
%            MM                                             
%  ,6"Yb.    MM  ,p6"bo   ,6"Yb.  M"""MMV  ,6"Yb.  `7Mb,od8 
% 8)   MM    MM 6M'  OO  8)   MM  '  AMV  8)   MM    MM' "' 
%  ,pm9MM    MM 8M        ,pm9MM    AMV    ,pm9MM    MM     
% 8M   MM    MM YM.    , 8M   MM   AMV  , 8M   MM    MM     
% `Moo9^Yo..JMML.YMbmd'  `Moo9^Yo.AMMmmmM `Moo9^Yo..JMML.   
% 
% 
% Free and Open-Source template for academic works
% https://github.com/dpmj/alcazar

\newpage

\clearpage
\cleardoublepage
\phantomsection

\pagestyle{plain}
\pagenumbering{gobble}

\phantomsection
\addcontentsline{toc}{chapter}{Abstract}

\begin{figure}
    \centering
    \includegraphics[width=50px]{figures/logos/k8s-bw-2.png}
    \label{fig:abstract:kubernetes}
\end{figure}

\begin{center}
    \large \textbf{\thesisTitle}
\end{center}

{\noindent \textbf{\textsc{Keywords}}}

{\noindent \thesisKeywords}\\


{\noindent \textbf{\textsc{Abstract}}}

\noindent Il \glsname{ml} è una delle branche dell'informatica con le più complesse esigenze computazionali. La difficoltà intrinseca del realizzare modelli di machine learning si amplifica ulteriormente quando sorge la necessità - pressante e inevitabile - di traslare le fasi di sviluppo e di deployment ad un contesto distribuito. Grazie alla sua natura altamente parallelizzabile, i carichi di machine learning sono candidati ideale per i sistemi di orchestrazione di container, che fanno leva sulla scalabilità verticale per assicurare virtualmente un'infinita capacità computazionale. 

Nell'ottica di questo lavoro di tesi, due modelli di machine learning dediti all'analisi genomica e realizzati con un paradigma monolitico sono stati sottoposti ad un processo di frammentazione in microservizi containerizzati. Avendo rilevato un sostanziale miglioramento delle performance del modello grazie alla sua nuova architettura, ci si è curati di trasportare la medesima su un cluster Kubernetes in grado di poterne gestire agilmente e con resilienza il ciclo di vita, dall'allenamento alla predizione, orchestrando queste fasi medianti Kubeflow. I risultati ottenuti trovano largo impiego sia su ecosistemi \textit{cloud-native} che \textit{bare metal}.

Al fine ultimo di tarare l'effettiva possibilità di trasferire in produzione quanto prodotto, è stata condotta un'analisi ad ampio spettro delle possibili vulnerabilità dell'architettura, immaginando molteplici scenari di attacco da parte di un avversario sulla rete. Quest'analisi ha permesso di valutare l'effettiva resilienza del sistema, e di proporre una serie di contromisure atte a mitigare i rischi individuati. Contestualmente, una serie di contributi sperimentali legati alla produzione di immagini Docker e alla loro distribuzione su un cluster Kubernetes sono stati proposti e discussi.

Lo studio condotto sottolinea il ruolo critico di Kubernetes nei sistemi ad alta complessità come quello presentato, ed evidenzia come il paradigma MLOps sia la modalità più organica e competitiva per sfruttare pienamente il potenziale del machine learning mediante l'adozione di architetture scalabili e resilienti con Kubernetes.

Uno spettro di possibili sviluppi futuri è stato osservato contestualmente all'attività progettuale: ad esempio, tecnologie come \glsname{mxnet}, \glsname{ray} o \glsname{mlflow} potrebbero costituire alternative valide a Kubeflow; inoltre, l'integrazione di strumenti di data governance come \glsname{fybrik} consentirebbe l'impiego di dataset contenenti PII, ampliando ulteriormente il campo di applicazione del sistema.